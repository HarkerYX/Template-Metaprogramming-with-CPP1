\hspace*{\fill} \\ %插入空行
\noindent
\textbf{习题 1}

什么时候执行名称查找?

\hspace*{\fill} \\ %插入空行
\noindent
\textbf{参考答案}

模板实例化依赖名称(依赖模板形参的类型或值)时执行名称查找,在模板定义非依赖名称(不依赖模板形参的名称)时执行名称查找。


\hspace*{\fill} \\ %插入空行
\noindent
\textbf{习题 2}

什么是推导指南?

\hspace*{\fill} \\ %插入空行
\noindent
\textbf{参考答案}

推导指南是一种告诉编译器如何执行类模板参数推导的机制。推导指南是表示虚构类类型的构造函数签名的虚构函数模板。若重载解析在构造的虚拟函数模板集上失败,则程序是格式错误的,并生成一个错误。否则,所选函数模板特化的返回类型,将成为推导类模板的特化。

\hspace*{\fill} \\ %插入空行
\noindent
\textbf{习题 3}

什么是转发引用?

\hspace*{\fill} \\ %插入空行
\noindent
\textbf{参考答案}

转发引用(也称为通用引用)是模板中的一种引用,若右值作为参数传递,则表现为右值引用;若左值作为参数传递,则表现为左值引用。转发引用必须具有T\&\&形式,例如template <typename T> void f(T\&\&)。像T const \&\&或std::vector<T>\&\&这样的格式并不表示转发引用,而是正常的右值引用。

\hspace*{\fill} \\ %插入空行
\noindent
\textbf{习题 4}

decltype是用来做什么的?

\hspace*{\fill} \\ %插入空行
\noindent
\textbf{参考答案}

decltype说明符是一个类型说明符,返回表达式的类型,通常在模板中与auto说明符一起使用,以声明依赖于模板参数的函数模板的返回类型,或者包装另一个函数并返回执行包装函数中结果函数的返回类型。

\hspace*{\fill} \\ %插入空行
\noindent
\textbf{习题 5}

std::declval是用来做什么的?

\hspace*{\fill} \\ %插入空行
\noindent
\textbf{参考答案}

std::declval是<utility>头文件中的工具函数模板,可将右值引用添加到其类型模板参数中。期只能用于未求值的上下文中(仅编译时上下文中,在运行时不求值),目的是帮助对没有默认构造函数,或由于私有或受保护而不能访问的类型进行依赖类型求值。












