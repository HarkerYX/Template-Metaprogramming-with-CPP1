\hspace*{\fill} \\ %插入空行
\noindent
\textbf{习题 1}

什么是约束,什么是概念?

\hspace*{\fill} \\ %插入空行
\noindent
\textbf{参考答案}

约束是强加在模板参数上的要求。概念是一个或多个约束的命名集。


\hspace*{\fill} \\ %插入空行
\noindent
\textbf{习题 2}

什么是requires子句和requires表达式?

\hspace*{\fill} \\ %插入空行
\noindent
\textbf{参考答案}

requires子句是一个构造,允许在模板参数或函数声明上指定约束。这个构造由requires关键字和一个编译时布尔表达式组成。require子句影响函数的行为,只有当布尔表达式为true时,才包括它进行重载解析。另一方面,requires表达式有requires(参数列表)表达式,其中parameters-list可选,其目的是验证某些表达式是否格式良好,而不会产生任何副作用或影响函数的行为。requires表达式可以与requires子句一起使用。不过,推荐首选命名概念,主要是从可读性的角度考虑。

\hspace*{\fill} \\ %插入空行
\noindent
\textbf{习题 3}

requires表达式的类别是什么?

\hspace*{\fill} \\ %插入空行
\noindent
\textbf{参考答案}

需求表达式有四类:简单需求、类型需求、复合需求和嵌套需求。

\hspace*{\fill} \\ %插入空行
\noindent
\textbf{习题 4}

约束如何影响重载解析中模板的顺序?

\hspace*{\fill} \\ %插入空行
\noindent
\textbf{参考答案}

函数的约束影响重载解析他们的顺序。当多个重载与参数集匹配时,将选择约束更强的重载。但请记住,类型特征(或布尔表达式)和概念的约束在语义上不相同。

\hspace*{\fill} \\ %插入空行
\noindent
\textbf{习题 5}

简写函数模板是什么?

\hspace*{\fill} \\ %插入空行
\noindent
\textbf{参考答案}

缩写函数模板是C++20引入的一个新特性,为函数模板提供了简化的语法。自动说明符可用于定义函数形参,模板语法可跳过。编译器会自动从缩写的函数模板生成一个函数模板。可以使用概念约束这些函数,因此可以对模板参数施加约束。












