
\hspace*{\fill} \\ %插入空行
\noindent
\textbf{Question 1}

Why do we need templates? What advantages do they provide?

\hspace*{\fill} \\ %插入空行
\noindent
\textbf{Answer}

There are several benefits to using templates: they help us avoid writing repetitive code, they foster the creation of generic libraries, and they can help us write less and better code.

\hspace*{\fill} \\ %插入空行
\noindent
\textbf{Question 2}

How do you call a function that is a template? What about a class that is a template?

\hspace*{\fill} \\ %插入空行
\noindent
\textbf{Answer}

A function that is a template is called a function template. Similarly, a class that is a template is called a class template.

\hspace*{\fill} \\ %插入空行
\noindent
\textbf{Question 3}

How many kinds of template parameters exist and what are they?

\hspace*{\fill} \\ %插入空行
\noindent
\textbf{Answer}

There are three kinds of template parameters: type template parameters, non-type template parameters, and template template parameters.

\hspace*{\fill} \\ %插入空行
\noindent
\textbf{Question 4}

What is partial specialization? What about full specialization?

\hspace*{\fill} \\ %插入空行
\noindent
\textbf{Answer}

Specialization is the technique of providing an alternative implementation for a template, called the primary template. Partial specialization is an alternative implementation provided for only some of the template parameters. A full specialization is an alternative implementation when arguments are provided for all the template parameters.

\hspace*{\fill} \\ %插入空行
\noindent
\textbf{Question 5}

What are the main disadvantages of using templates?

\hspace*{\fill} \\ %插入空行
\noindent
\textbf{Answer}

The main disadvantages of using templates include the following: complex and cumbersome syntax, compiler errors that are often long and hard to read and understand, and increased compilation times.



