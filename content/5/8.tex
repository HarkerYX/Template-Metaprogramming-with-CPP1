\hspace*{\fill} \\ %插入空行
\noindent
\textbf{习题 1}

标准库中的序列容器是什么?

\hspace*{\fill} \\ %插入空行
\noindent
\textbf{参考答案}

C++标准库中的序列容器有std::vector、std::deque、std::list、std::array和std::forward\_list。


\hspace*{\fill} \\ %插入空行
\noindent
\textbf{习题 2}

标准容器中定义的公共成员函数是什么?

\hspace*{\fill} \\ %插入空行
\noindent
\textbf{参考答案}

标准库中为大多数容器定义的成员函数是size(std::forward\_list中不存在)、empty、clear(std::array、std::stack、std::queue和std::priority\_queue中不存在)、swap、begin和end。

\hspace*{\fill} \\ %插入空行
\noindent
\textbf{习题 3}

什么是迭代器,存在多少个类别?

\hspace*{\fill} \\ %插入空行
\noindent
\textbf{参考答案}

迭代器是一种抽象,能够以通用的方式访问容器的元素,而不必知道每个容器的实现细节。迭代器是编写通用算法的关键。C++中有六类迭代器:输入、向前、双向、随机访问、连续(至C++17)和输出。

\hspace*{\fill} \\ %插入空行
\noindent
\textbf{习题 4}

随机迭代器支持哪些操作?

\hspace*{\fill} \\ %插入空行
\noindent
\textbf{参考答案}

随机迭代器必须支持以下操作(除了输入迭代器、正向迭代器和双向迭代器所需的操作之外):+和-算术运算符、不等式比较(与其他迭代器)、复合赋值和偏移解引用操作符。

\hspace*{\fill} \\ %插入空行
\noindent
\textbf{习题 5}

范围访问函数是什么?

\hspace*{\fill} \\ %插入空行
\noindent
\textbf{参考答案}

范围访问函数是非成员函数,提供统一的方式来访问容器、数组和std::initializer\_list类的数据或属性。这些函数包括std::size/std::ssize、std::empty、std::data、std::begin和std::end。












