折叠表达式是一种包含参数包的表达式,它将参数包中的元素折叠(或减少)到二进制运算符上。为了了解其工作原理,我们可以来看几个例子。之前,我们实现了一个名为sum的变量函数模板,返回所有提供参数的和。方便起见,我们将在这里再次展示:

\begin{lstlisting}[style=styleCXX]
template <typename T>
T sum(T a)
{
	return a;
}

template <typename T, typename... Args>
T sum(T a, Args... args)
{
	return a + sum(args...);
}
\end{lstlisting}

使用折叠表达式,需要两次重载的实现可以简化为以下形式:

\begin{lstlisting}[style=styleCXX]
template <typename... T>
int sum(T... args)
{
	return (... + args);
}
\end{lstlisting}

不再需要重载函数了。表达(…+ args)表示折叠表达式,计算后变为((((arg0 + arg1) + arg2) +…)+ argN)。括号是折叠表达式的一部分,可以使用这个新的实现,就像使用最初的实现一样:

\begin{lstlisting}[style=styleCXX]
int main()
{
	std::cout << sum(1) << '\n';
	std::cout << sum(1,2) << '\n';
	std::cout << sum(1,2,3,4,5) << '\n';
}
\end{lstlisting}

折叠方式有四种:

\begin{table}[H]
\centering
	\begin{tabular}{|l|l|l|}
		\hline
		\textbf{折叠方式}    & \textbf{语法} & \textbf{展开方式}                  \\ \hline
		一元右折叠 & (pack op ...)   & (arg1 op (... op (argN-1 op argN))) \\ \hline
		一元左折叠  & (... op pack)   & (((arg1 op arg2) op ...) op argN)   \\ \hline
		二元右折叠 &
		\begin{tabular}[c]{@{}l@{}}(pack op ... op\\ init)\end{tabular} &
		\begin{tabular}[c]{@{}l@{}}(arg1 op (... op (argN-1 op (argN\\ op init))))\end{tabular} \\ \hline
		二元左折叠 &
		\begin{tabular}[c]{@{}l@{}}(init op ... op\\ pack)\end{tabular} &
		\begin{tabular}[c]{@{}l@{}}((((init op arg1) op arg2) op ...)\\ op argN)\end{tabular} \\ \hline
	\end{tabular}
\end{table}

\begin{center}
表 3.1
\end{center}

在该表中,使用了以下名词:

\begin{itemize}
\item
pack是一个包含未展开形参包的表达式,arg1、arg2、argN-1和argN是这个包中包含的参数。

\item
op下面是一个二元操作符 : +, -, *, /, \%, \^, \&, |, =, <, >, <{}<, >{}>, +=, -=, *=, /=, \%=, \^=, \&=, |=, <{}<=, >{}>=, ==, !=, <=, >=, \&\&, ||, ,(逗号表达式), .*, ->*.

\item
init不包含未展开参数包的表达式。
\end{itemize}

一元折叠中,若参数包不包含任何元素,则只允许使用某些操作符。下表列出了这些值,以及空参数包的值:

\begin{table}[H]
\centering
	\begin{tabular}{|l|l|}
		\hline
		\textbf{操作符}  & \textbf{空参数包的值} \\ \hline
		\&\& (逻辑 AND) & true                             \\ \hline
		|| (逻辑 OR)    & false                            \\ \hline
		, (逗号操作符) & void()                           \\ \hline
	\end{tabular}
\end{table}

\begin{center}
表 3.2
\end{center}

一元折叠和二元折叠的不同之处在于初始化值的使用,初始化值只适用于二元折叠。二元折叠将二元操作符重复两次(必须是同一个操作符)。通过包含初始化值,可以将可变函数模板和从使用一元右折叠表达式转换为使用二元右折叠表达式。这里有一个例子:

\begin{lstlisting}[style=styleCXX]
template <typename... T>
int sum_from_zero(T... args)
{
	return (0 + ... + args);
}
\end{lstlisting}

感觉sum和sum\_from\_zero的函数模板没有区别,但事实并非如此。让我们考虑以下情况:

\begin{lstlisting}[style=styleCXX]
int s1 = sum(); // error
int s2 = sum_from_zero(); // OK
\end{lstlisting}

调用不带参数的sum将产生编译器错误,因为一元折叠表达式(加法运算符)必须有非空展开。然而,二进制折叠表达式没有这个问题,因此调用sum\_from\_zero而不带参数可以工作,函数将返回0。

这两个使用sum和sum\_from\_zero的例子中,参数包args直接出现在折叠表达式中。但它可以是表达式的一部分,只要没有展开。如下例所示:

\begin{lstlisting}[style=styleCXX]
template <typename... T>
void printl(T... args)
{
	(..., (std::cout << args)) << '\n';
}

template <typename... T>
void printr(T... args)
{
	((std::cout << args), ...) << '\n';
}
\end{lstlisting}

这里,参数包args是(std::cout <{}< args)表达式的一部分。这不是一个折叠表达式。折叠表达式是((std::cout <{}< args),…)。这是逗号运算符上的一元左折叠。printl和printr函数可以像下面这样使用:

\begin{lstlisting}[style=styleCXX]
printl('d', 'o', 'g'); // dog
printr('d', 'o', 'g'); // dog
\end{lstlisting}

这两种情况下,输出到控制台的文本都是dog。这是因为一元左折叠扩展为(((std::cout <{}< 'd'), std::cout <{}< 'o'), << std::cout <{}< 'g'),一元右折叠扩展为(std::cout <{}< 'd', (std::cout <{}< 'o', (std::cout <{}< 'g')),因为由逗号分隔的一对表达式从左到右求值,所以这两种计算方式相同,内置逗号操作符就是这样。对于重载逗号操作符的类型,其行为取决于如何重载该操作符。然而,重载逗号操作符的情况非常少(例如简化多维数组索引),如Boost库。赋值和SOCI会重载逗号运算符,但通常情况下,最好不要去重载逗号操作符。

来看一下在折叠表达式中的表达式中使用参数包的另一个示例。下面的可变参数函数模板,会将多个值插入到std::vector的末尾:

\begin{lstlisting}[style=styleCXX]
template<typename T, typename... Args>
void push_back_many(std::vector<T>& v, Args&&... args)
{
	(v.push_back(args), ...);
}

push_back_many(v, 1, 2, 3, 4, 5); // v = {1, 2, 3, 4, 5}
\end{lstlisting}

参数包args与v.push\_back(args)表达式一起使用,该表达式折叠在逗号操作符上。所以,一元左折叠表达式为(v.push\_back(args),…)。

与使用递归实现可变参数模板相比,折叠表达式有几个优点:

\begin{itemize}
\item
代码更少、更简单。

\item
更快的编译时间,更少的模板实例化。

\item
可能更快的代码,因为多个函数调用替换为单个表达式。然而,这一点在实践中有待验证,至少在启用优化时不是。我们已经看到编译器通过删除这些函数来调用优化代码。
\end{itemize}

现在我们已经了解了如何创建可变参数函数模板、可变参数类模板,以及如何使用折叠表达式,接下来将讨论其他类型的可变参数模板:别名模板和变量模板。




















