
使用模板之前,了解使用模板的优缺点相当重要。

先看看优点:

\begin{itemize}
\item
避免编写重复的代码。

\item
提供算法和类型的泛型库的创建,例如标准C++库(有时\textit{错误地}称为STL),无论是何类型,都可以在许多应用程序中使用。

\item
使用模板可以得到少而优的代码。例如,使用标准库中的算法可以帮助编写更少的代码,这些代码可能更容易理解和维护,并且可能更健壮(这些算法的开发和测试投入了相当多的精力)。
\end{itemize}

再来看看缺点:

\begin{itemize}
\item
语法非常复杂,但只要进行过一些实践,这应该不会对模板的开发和使用造成影响。

\item
与模板代码相关的编译器错误通常很长而且神秘,很难确定其原因。新版本的C++编译器已经简化了这类错误(仍然是一个重要的问题),C++20标准中的概念就可以看做为是一种尝试(包括帮助为编译错误提供更好的诊断信息)。

\item
增加了编译时间,因为其完全在头文件中实现。每当对模板进行更改时,包含该头文件的所有翻译单元都必须重新编译。

\item
模板库是作为一个或多个头文件的集合,必须与使用其代码一起编译。

\item
头文件中实现模板的另一个缺点是没有信息隐藏,所有人都可以在头文件阅读整个模板代码。标准库开发人员经常使用具有诸如detail或details等命名空间来包含标准库内部的代码,但这些代码是标准库的使用者不应该使用的。

\item
由于编译器没有实例化未使用的代码,因此这部分代码可能更难验证。因此,在编写单元测试时,必须确保良好的代码覆盖率,对于库来说尤是如此。
\end{itemize}

虽然缺点有点多,但使用模板并不是一件坏事或应该避免的事情。相反,模板是C++语言的强大特性。模板并不总能被正确理解,有时还会误用或过度使用。然而,明智地使用模板具有毋庸置疑的优势。这本书将提供一中更好的理解模板及其使用的方式。







