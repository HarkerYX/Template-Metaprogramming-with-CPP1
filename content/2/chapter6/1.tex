概念可以让代码有更好的可读性和错误消息。在了解如何使用概念之前,先来回顾一下之前的例子:

\begin{lstlisting}[style=styleCXX]
template <typename T>
T add(T const a, T const b)
{
	return a + b;
}
\end{lstlisting}

这个简单的函数模板接受两个参数,并返回它们的和。实际上,返回的不是和,而是对两个参数应用加号运算符后的结果。用户定义的类型可以重载此操作符并执行某些特定的操作。术语“和”只有在讨论数学类型时才有意义,例如整型、浮点类型、std::complex类型、矩阵类型、向量类型等。

对于字符串类型,加号操作符可以表示连接。对于大多数类型来说,重载根本没有意义。因此,只看函数的声明,而不检查它的主体,不能真正说这个函数可以接受什么作为输入,以及这个函数是做什么的。可以这样调用这个函数:

\begin{lstlisting}[style=styleCXX]
add(42, 1); // [1]
add(42.0, 1.0); // [2]
add("42"s, "1"s); // [3]
add("42", "1"); // [4] error: cannot add two pointers
\end{lstlisting}

前三个调用都很好。第一个调用添加两个整数,第二个调用添加两个double值,第三个调用连接两个std::string对象。但第四次调用将产生编译器错误,因为const char *替换了T类型模板参数,并且指针类型没有重载加号操作符。

这个add函数模板的目的是只允许传递算术类型的值,即整型和浮点型。C++20之前,可以通过几种方式实现这一点。一种方法是使用std::enable\_if和SFINAE,就像在前一章看到的那样:

\begin{lstlisting}[style=styleCXX]
template <typename T,
	typename = typename std::enable_if_t
		<std::is_arithmetic_v<T>>>
T add(T const a, T const b)
{
	return a + b;
}
\end{lstlisting}

首先,可读性下降了。第二,类模板参数难以阅读,需要对模板有很好的了解才能理解。但这一次,标记为[3]和[4]的行上的调用都产生了编译器错误。不同的编译器会产生不同的错误消息。下面是三个主要编译器的代码:

\begin{itemize}
\item
VC++ 17:

\begin{tcblisting}{commandshell={}}
error C2672: 'add': no matching overloaded function found
error C2783: 'T add(const T,const T)': could not deduce
template argument for '<unnamed-symbol>'
\end{tcblisting}

\item
GCC 12:
\begin{tcblisting}{commandshell={}}
prog.cc: In function 'int main()':
prog.cc:15:8: error: no matching function for call
to 'add(std::__cxx11::basic_string<char>, std::__
cxx11::basic_string<char>)'
15 |         add("42"s, "1"s);
   |       ~~~^~~~~~~~~~~~~
prog.cc:6:6: note: candidate: 'template<class T, class> T
add(T, T)'
6 | T add(T const a, T const b)
  |     ^~~
prog.cc:6:6: note: template argument deduction/
substitution failed:
In file included from /opt/wandbox/gcc-head/include/
c++/12.0.0/bits/move.h:57,
                 from /opt/wandbox/gcc-head/include/
c++/12.0.0/bits/nested_exception.h:40,
                 from /opt/wandbox/gcc-head/include/
c++/12.0.0/exception:154,
                 from /opt/wandbox/gcc-head/include/
c++/12.0.0/ios:39,
                 from /opt/wandbox/gcc-head/include/
c++/12.0.0/ostream:38,
                 from /opt/wandbox/gcc-head/include/
c++/12.0.0/iostream:39,
                 from prog.cc:1:
/opt/wandbox/gcc-head/include/c++/12.0.0/type_traits: In
substitution of 'template<bool _Cond, class _Tp> using
enable_if_t = typename std::enable_if::type [with bool _
Cond = false; _Tp = void]':
prog.cc:5:14: required from here
/opt/wandbox/gcc-head/include/c++/12.0.0/type_traits:2603:11:
error: no type named 'type' in 'struct std::enable_if<false, 
void>'
2603 | using enable_if_t = typename enable_if<_Cond, _Tp>::type;
     |            ^~~~~~~~~~~
\end{tcblisting}

\item
Clang 13:
\begin{tcblisting}{commandshell={}}
prog.cc:15:5: error: no matching function for call to
'add'
    add("42"s, "1"s);
    ^~~
prog.cc:6:6: note: candidate template ignored:
requirement 'std::is_arithmetic_v<std::string>' was not
satisfied [with T = std::string]
   T add(T const a, T const b)
      ^
\end{tcblisting}
\end{itemize}

GCC中的错误消息非常冗长,VC++没有说明匹配模板参数失败的原因是什么。可以说,Clang在提供可理解的错误消息方面做得更好。

C++20之前,定义此函数限制的另一种方法是使用static\_assert:

\begin{lstlisting}[style=styleCXX]
template <typename T>
T add(T const a, T const b)
{
	static_assert(std::is_arithmetic_v<T>,
				  "Arithmetic type required");
	return a + b;
}
\end{lstlisting}

这个实现中,我们回到了最初的问题,即仅通过查看函数的声明,不知道它会接受什么样的参数,没有任何限制。另一方面,错误信息如下所示:

\begin{itemize}
\item
VC++ 17:

\begin{tcblisting}{commandshell={}}
error C2338: Arithmetic type required
main.cpp(157): message : see reference to function
template instantiation 'T add<std::string>(const T,const
T)' being compiled
     with
     [
         T=std::string
     ]
\end{tcblisting}

\item
GCC 12:

\begin{tcblisting}{commandshell={}}
prog.cc: In instantiation of 'T add(T, T) [with T =
std::__cxx11::basic_string<char>]':
prog.cc:15:8: required from here
prog.cc:7:24: error: static assertion failed: Arithmetic
type required
    7 | static_assert(std::is_arithmetic_v<T>,
"Arithmetic type required");
      |                         ~~~~~^~~~~~~~~~~~~~~~~~
prog.cc:7:24: note: 'std::is_arithmetic_v<std::__
cxx11::basic_string<char> >' evaluates to false
\end{tcblisting}

\item
Clang 13:

\begin{tcblisting}{commandshell={}}
prog.cc:7:5: error: static_assert failed due to
requirement 'std::is_arithmetic_v<std::string>'
"Arithmetic type required"
     static_assert(std::is_arithmetic_v<T>, "Arithmetic
type required");
     ^ ~~~~~~~~~~~~~~~~~~~~~~~
prog.cc:15:5: note: in instantiation of function template
specialization 'add<std::string>' requested here
    add("42"s, "1"s);
    ^
\end{tcblisting}
\end{itemize}

不管编译器是什么,使用static\_assert会导致接收到类似的错误消息。

C++20中,可以通过使用约束来改进这两个方面(可读性和错误消息)。这些是通过新关键字requires引入的,如下所示:

\begin{lstlisting}[style=styleCXX]
template <typename T>
requires std::is_arithmetic_v<T>
T add(T const a, T const b)
{
	return a + b;
}
\end{lstlisting}

requires关键字引入了一个称为requires子句,定义了模板参数的约束。实际上,有两种可供选择的语法:一种是requires子句跟在模板参数列表后面,如前所述;另一种是requires子句跟在函数声明后面,如下所示:

\begin{lstlisting}[style=styleCXX]
template <typename T>
T add(T const a, T const b)
requires std::is_arithmetic_v<T>
{
	return a + b;
}
\end{lstlisting}

这两种语法可以根据个人喜好进行选择,但这两种情况,可读性都比C++20之前的实现好得多。通过阅读声明,就知道T类型模板参数必须是算术类型。此外,函数只是两个数字的相加,不需要看定义就能知道。当调用带有无效参数的函数时,错误消息的情况:

\begin{itemize}
\item
VC++ 17:

\begin{tcblisting}{commandshell={}}
error C2672: 'add': no matching overloaded function found
error C7602: 'add': the associated constraints are not
satisfied
\end{tcblisting}

\item
GCC 12:

\begin{tcblisting}{commandshell={}}
prog.cc: In function 'int main()':
prog.cc:15:8: error: no matching function for call
to 'add(std::__cxx11::basic_string<char>, std::__cxx11::
basic_string<char>)'
  15 |    add("42"s, "1"s);
     |     ~~~^~~~~~~~~~~~~
prog.cc:6:6: note: candidate: 'template<class T> 
requires is_arithmetic_v<T> T add(T, T)'
  6  |    T add(T const a, T const b)
     |      ^~~
prog.cc:6:6: note: template argument deduction/substitution
 failed:
prog.cc:6:6: note: constraints not satisfied
prog.cc: In substitution of 'template<class
T> requires is_arithmetic_v<T> T add(T, T) [with T =
std::__cxx11::basic_string<char>]':
prog.cc:15:8: required from here
prog.cc:6:6: required by the constraints of
'template<class T> requires is_arithmetic_v<T> T add(T,
T)'
prog.cc:5:15: note: the expression 'is_arithmetic_v<T>
[with T = std::__cxx11::basic_string<char, std::char_
traits<char>, std::allocator<char> >]' evaluated to 'false'
    5 | requires std::is_arithmetic_v<T>
      |               ~~~~~^~~~~~~~~~~~~~~~~~
\end{tcblisting}

\item
Clang 13:

\begin{tcblisting}{commandshell={}}
prog.cc:15:5: error: no matching function for call to
'add'
add("42"s, "1"s);
^~~
prog.cc:6:6: note: candidate template ignored:
constraints not satisfied [with T = std::string]
T add(T const a, T const b)
^
prog.cc:5:10: note: because 'std::is_arithmetic_
v<std::string>' evaluated to false
requires std::is_arithmetic_v<T>
^
\end{tcblisting}
\end{itemize}

错误消息遵循相同的模式:GCC太冗长,VC++缺少基本信息(未满足的约束),而Clang更简洁,更好地定位错误的原因。总的来说,错误消息看起来有了改进,但仍有改进空间。

约束是在编译时求值为true或false的谓词。前一个例子中使用的表达式std::is\_arithmetic\_v<T>,只是使用了一个标准类型特征(前一章中看到过),这些不同类型的表达式都可以在约束中使用。

下一节中,我们将讨论如何定义和使用命名约束。
























