约束是在使用它们的地方定义的匿名谓词。许多约束是通用的,可以在多个地方使用。以下是类似于add函数的函数示例,这个函数执行算术乘法:

\begin{lstlisting}[style=styleCXX]
template <typename T>
requires std::is_arithmetic_v<T>
T mul(T const a, T const b)
{
	return a * b;
}
\end{lstlisting}

与add函数相同的requires子句也出现在这里。为了避免这种重复的代码,可以定义一个可以在多个地方重用的名称约束,命名约束也称为概念。概念用新关键字concept和模板语法定义。这里有一个例子:

\begin{lstlisting}[style=styleCXX]
template<typename T>
concept arithmetic = std::is_arithmetic_v<T>;
\end{lstlisting}

即使赋值为布尔值,概念名也不应该包含动词。它们表示需求,并用作模板参数的属性或限定符,所以应该使用arithmetic、copyable、serializable、container等名称,而不是is\_arithmetic、is\_copyable、is\_serializable和is\_container。前面定义的arithmetic概念可以这样使用:

\begin{lstlisting}[style=styleCXX]
template <arithmetic T>
T add(T const a, T const b) { return a + b; }

template <arithmetic T>
T mul(T const a, T const b) { return a * b; }
\end{lstlisting}

这里直接使用了这个概念,而不是typename关键字。它用arithmetic来限定T类型,只有满足此要求的类型才能用作模板参数。相同的算术概念可以用不同的语法定义,如下所示:

\begin{lstlisting}[style=styleCXX]
template<typename T>
concept arithmetic = requires { std::is_arithmetic_v<T>; };
\end{lstlisting}

这使用了requires表达式,其使用大括号分支\{\},而requires子句没有。requires表达式可以包含不同类型的需求序列:简单需求、类型需求、复合需求和嵌套需求。这里看到的是一个简单的需求。为了定义这个特定的概念,这种语法更加复杂,但具有相同的最终效果。但在某些情况下,确实需要复杂的要求。

考虑这样一种情况,想要定义一个模板时,只接受容器类型作为参数。在概念出现之前,这个问题可以通过类型特征和SFINAE或static\_assert来解决,就像在本章开头看到的那样。不过,容器类型实际上不容易定义,可以基于标准容器的一些属性来实现:

\begin{itemize}
\item
其成员类型有value\_type、size\_type、allocator\_type、iterator和const\_iterator。

\item
具有成员函数size,该函数返回容器中元素的数量。

\item
具有成员函数begin/end和cbegin/cend,返回指向容器中第一个和倒数一个元素的迭代器和常量迭代器。
\end{itemize}

根据第5章的知识,可以这样定义is\_container类型特征:

\begin{lstlisting}[style=styleCXX]
template <typename T, typename U = void>
struct is_container : std::false_type {};

template <typename T>
struct is_container<T,
	std::void_t<typename T::value_type,
				typename T::size_type,
				typename T::allocator_type,
				typename T::iterator,
				typename T::const_iterator,
				decltype(std::declval<T>().size()),
				decltype(std::declval<T>().begin()),
				decltype(std::declval<T>().end()),
				decltype(std::declval<T>().cbegin()),
				decltype(std::declval<T>().cend())>>
	: std::true_type{};

template <typename T, typename U = void>
constexpr bool is_container_v = is_container<T, U>::value;
\end{lstlisting}

可以在static\_assert的帮助下验证类型特征是否正确地标识了容器类型:

\begin{lstlisting}[style=styleCXX]
struct foo {};

static_assert(!is_container_v<foo>);
static_assert(is_container_v<std::vector<foo>>);
\end{lstlisting}

概念使得编写这样的模板约束更加容易,可以使用概念语法和requires表达式来定义以下内容:

\begin{lstlisting}[style=styleCXX]
template <typename T>
concept container = requires(T t)
{
	typename T::value_type;
	typename T::size_type;
	typename T::allocator_type;
	typename T::iterator;
	typename T::const_iterator;
	t.size();
	t.begin();
	t.end();
	t.cbegin();
	t.cend();
};
\end{lstlisting}

这个定义更短,可读性更强。既使用简单需求,如t.size(),也使用类型需求,如typename T::value\_type。可以用前面看到的方式来约束模板参数,但也可以与static\_assert一起使用(约束计算为编译时布尔值):

\begin{lstlisting}[style=styleCXX]
struct foo{};

static_assert(!container<foo>);
static_assert(container<std::vector<foo>>);

template <container C>
void process(C&& c) {}
\end{lstlisting}

下一节中,将深入探讨可以在requires表达式中使用的各种需求。


























