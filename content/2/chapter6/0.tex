The C++20 standard provides a series of significant improvements to template metaprogramming with concepts and constraints. A constraint is a modern way to define requirements on template parameters. A concept is a set of named constraints. Concepts provide several benefits to the traditional way of writing templates, mainly improved readability of code, better diagnostics, and reduced compilation times.

In this chapter, we will address the following topics:

\begin{itemize}
\item
Understanding the need for concepts

\item
Defining concepts

\item
Exploring requires expressions

\item
Composing constraints

\item
Learning about the ordering of templates with constraints

\item
Constraining non-template member functions

\item
Constraining class templates

\item
Constraining variable templates and template aliases

\item
Learning more ways to specify constraints

\item
Using concepts to constrain auto parameters

\item
Exploring the standard concepts library
\end{itemize}

By the end of this chapter, you will have a good understanding of the C++20 concepts, and an overview of what concepts the standard library provides.

We will start the chapter by discussing what led to the development of concepts and what their main benefits are.


























