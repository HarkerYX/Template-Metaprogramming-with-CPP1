
标准库提供了一系列类型特征,用于查询类型的属性以及对类型执行转换。这些类型特征可以在<type\_ traits>头文件中作为类型支持库使用。类型特征有以下几种:

\begin{itemize}
\item
查询类型类别(主或复合类型)

\item
查询类型属性

\item
查询支持的操作

\item
查询类型关系

\item
修改cv说明符、引用、指针或符号

\item
各种转换
\end{itemize}

尽管研究每一种类型的特征超出了本书的范畴,但我们将探讨所有这些类别,看看它们包含什么。下面的小节中,将列出构成这些类别的类型特征(或大部分类型特征)。这些列表以及关于每种类型特征的详细信息可以在C++标准中找到(参见本章末尾的扩展阅读,以获得标准草案版本的链接)或cppreference.com网站\url{https://en.cppreference.com/w/cpp/header/type_traits}(许可链接:\url{http://creativecommons.org/licenses/by-sa/3.0/})。

\subsubsubsection{5.5.1\hspace{0.2cm}查询类型类别}

之前已经使用了几个类型特征,例如std::is\_integral,std::is\_floating\_point,以及std::is\_arithmetic。这些只是用于查询主类型和复合类型类别的一些标准类型特征。下表列出了所有这些类型的特征:

\begin{table}[H]
\centering
	\begin{tabular}{|l|l|}
		\hline
		\textbf{名称} &
		\textbf{描述} \\ \hline
		is\_void &
		检查类型是否为void类型。 \\ \hline
		is\_null\_pointer &
		检查一个类型是否为std::nullptr\_t类型。 \\ \hline
		is\_integral &
		\begin{tabular}[c]{@{}l@{}}检查类型是否为整型,包括有符号、无符号和cv限定变量。整型类型为:\\ ·bool, char, char8\_t(C++20), char16\_t, char32\_t, wchar\_t, short, \\ int, long和long long\\ ·任何扩展整数类型\end{tabular} \\ \hline
		is\_floating\_point &
		\begin{tabular}[c]{@{}l@{}}检查类型是否为浮点类型,包括cv限定变量。T可能的类型有\\ float,double和long double.\end{tabular}\\ \hline
		is\_array &
		检查类型是否为数组类型。 \\ \hline
		is\_enum &
		检查类型是否为枚举类型。 \\ \hline
		is\_union &
		检查类型是否为联合类型。 \\ \hline
		is\_class &
		检查类型是否是类类型,而不是联合类型。 \\ \hline
		is\_function &
		\begin{tabular}[c]{@{}l@{}}检查类型是否为函数类型。除了Lambda,重载调用操作符的类,\\ 指向函数的指针,而std::function类型除外。\end{tabular} \\ \hline
	\end{tabular}
\end{table}

\begin{table}[H]
\centering
	\begin{tabular}{|l|l|}
		\hline
		\textbf{名称} &
		\textbf{描述} \\ \hline	
		is\_pointer & 
		\begin{tabular}[c]{@{}l@{}}检查类型是指向对象的指针、指向函数的指针还是cv限定变量。\\ 这不包括指向成员对象的指针或指向成员函数的指针。\end{tabular} \\ \hline

		is\_member\_pointer &
		\begin{tabular}[c]{@{}l@{}}检查类型是指向非静态成员对象的指针,\\ 还是指向非静态成员函数的指针。\end{tabular} \\ \hline
		\begin{tabular}[c]{@{}l@{}}is\_member\_object\_\\ pointer\end{tabular} &
		检查类型是否为非静态成员对象指针。 \\ \hline
		\begin{tabular}[c]{@{}l@{}}is\_member\_function\_\\ pointer\end{tabular} &
		检查类型是否为非静态成员函数指针。 \\ \hline
		is\_lvalue\_reference &
		检查类型是否为左值引用类型。 \\ \hline
		is\_rvalue\_reference &
		检查类型是否为右值引用类型。 \\ \hline
		is\_reference &
		\begin{tabular}[c]{@{}l@{}}检查类型是否为引用类型,可以是左值引用类型,\\ 也可以是右值引用类型。\end{tabular} \\ \hline
		is\_fundamental &
		\begin{tabular}[c]{@{}l@{}}检查类型是否为基本类型。基本类型为算术类型, \\ void类型和std::nullptr\_t类型。.\end{tabular} \\ \hline
		is\_scalar &
		\begin{tabular}[c]{@{}l@{}}检查类型是标量类型还是标量类型的cv-限定符版本。 标量类型包括:\\ ·算术类型\\ ·指针类型\\ ·指向成员类型的指针\\ ·枚举类型\\ ·std::nullptr\_t\end{tabular} \\ \hline
		is\_object &
		\begin{tabular}[c]{@{}l@{}}检查类型是否为cv-qualifier版本的对象类型。\\ 对象类型不是函数类型、引用类型或void类型。\end{tabular} \\ \hline
		is\_compound &
		\begin{tabular}[c]{@{}l@{}}检查类型是复合类型还是复合类型的cv变体。\\ 复合类型不是基本类型,它们是\\ ·数组\\ ·函数\\ ·类\\ ·联合\\ ·对象指针和函数指针\\ ·成员对象指针和成员函数指针\\ ·引用\\ ·枚举\end{tabular} \\ \hline
	\end{tabular}
\end{table}

\begin{center}
表 5.1
\end{center}

这些类型特征在C++11中可用。从C++17开始,每个变量都有一个变量模板来简化对布尔成员value的访问。对于名称为is\_abc的类型特征,则存在名称为is\_abc\_v的变量模板。对于所有具有名为value的布尔成员的类型特征都是如此,这些变量的定义很简单。下面的代码片段显示了is\_arithmentc\_v变量模板的定义:

\begin{lstlisting}[style=styleCXX]
template< class T >
inline constexpr bool is_arithmetic_v =
	is_arithmetic<T>::value;
\end{lstlisting}

下面是使用这些类型特征的例子:

\begin{lstlisting}[style=styleCXX]
template <typename T>
std::string as_string(T value)
{
	if constexpr (std::is_null_pointer_v<T>)
		return "null";
	else if constexpr (std::is_arithmetic_v<T>)
		return std::to_string(value);
	else
		static_assert(always_false<T>);
}

std::cout << as_string(nullptr) << '\n'; // prints null
std::cout << as_string(true) << '\n'; // prints 1
std::cout << as_string('a') << '\n'; // prints a
std::cout << as_string(42) << '\n'; // prints 42
std::cout << as_string(42.0) << '\n'; // prints 42.000000
std::cout << as_string("42") << '\n'; // error
\end{lstlisting}

函数模板as\_string返回一个包含pass值作为参数的字符串。它只适用于算术类型,并且适用于nullptr\_t,它将为其返回值“null”。

聪明的读者一定注意到了static\_assert(always\_false<T>),并想知道这个always\_false<T>表达式到底是什么。它是一个bool类型的变量模板,计算结果为false。其定义简单如下:

\begin{lstlisting}[style=styleCXX]
template<class T>
constexpr bool always_false = std::false_type::value;
\end{lstlisting}

这是必需的,因为static\_assert(false)语句会使程序格式不正确。原因是它的条件不依赖于模板参数,而是求值为false。若模板中不能为constexpr if语句的子语句生成有效的特化,则程序是格式错误的(不需要诊断)。为了避免这种情况,static\_assert语句的条件必须依赖于模板参数。对于静态\_assert(always\_false<T>),编译器在模板实例化之前,不知道这将计算为true还是false。

我们探索的下一类类型特征,是查询类型的属性。

\subsubsubsection{5.5.2\hspace{0.2cm}查询类型属性}

能够查询类型属性的类型特征如下所示:

\begin{table}[H]
\centering
	\begin{tabular}{|l|l|l|}
		\hline
		\textbf{名称} & \textbf{C++版本} & \textbf{描述}                                                                                                      \\ \hline
		is\_const     & C++11                & \begin{tabular}[c]{@{}l@{}}是否为const限定(const或const volatile)。\end{tabular}       \\ \hline
		is\_volatile  & C++11                & \begin{tabular}[c]{@{}l@{}}是否为volatile限定(volatile或const volatile)。\end{tabular} \\ \hline
	\end{tabular}
\end{table}

\begin{table}[H]
\centering
	\begin{tabular}{|l|l|l|}
		\hline
		\textbf{名称} &
		\textbf{C++版本} &
		\textbf{描述} \\ \hline
		is\_trivial &
		C++11 &
		\begin{tabular}[c]{@{}l@{}}是普通类型,还是cv限定变量。以下是简单类型:\\  ·标量类型或标量类型的数组\\ ·具有简单的默认构造函数\\   或此类数组的简单的可复制类。\end{tabular} \\ \hline
		\begin{tabular}[c]{@{}l@{}}is\_trivially\_\\ copyable\end{tabular} &
		C++11 &
		\begin{tabular}[c]{@{}l@{}}是否可复制。 以下是可简单复制的类型:\\ ·标量类型或标量类型的数组\\ ·可简单复制的类或此类的数组\end{tabular} \\ \hline
		\begin{tabular}[c]{@{}l@{}}is\_standard\_\\ layout\end{tabular} &
		C++11 &
		\begin{tabular}[c]{@{}l@{}}是标准布局类型,还是cv限定变体\\ ·标量类型或标量类型的数组\\ ·标准布局类或此类的数组\end{tabular} \\ \hline
		is\_empty &
		C++11 &
		\begin{tabular}[c]{@{}l@{}}是否为空类型。空类型是一种类类型(不是联合),\\ 是否为空类型。空类型是一种类类型(不是联合),\\ 没有虚函数,没有虚基类,也没有非空基类。\end{tabular} \\ \hline
		is\_polymorphic &
		C++11 &
		\begin{tabular}[c]{@{}l@{}}检查是否为多态类型。多态类型是继承\\ 至少一个虚函数的类型(不是联合)。\end{tabular} \\ \hline
		is\_abstract &
		C++11 &
		\begin{tabular}[c]{@{}l@{}}检查是否为抽象类型。抽象类型是继承至少\\ 一个虚纯函数的类型(不是联合)。\end{tabular} \\ \hline
		is\_final &
		C++14 &
		\begin{tabular}[c]{@{}l@{}}检查是否是使用final说明符声明的类型。\end{tabular} \\ \hline
		is\_aggregate &
		C++17 &
		是否为聚合类型。 \\ \hline
		is\_signed &
		C++11 &
		\begin{tabular}[c]{@{}l@{}}检查是浮点类型,还是有符号整型。\end{tabular} \\ \hline
		is\_unsigned &
		C++11 &
		型是无符号整型,还是bool类型。 \\ \hline
		\begin{tabular}[c]{@{}l@{}}is\_bounded\_\\ array\end{tabular} &
		C++20 &
		\begin{tabular}[c]{@{}l@{}}检查是否为已知边界的数组类型(比如int{[}5{]})\end{tabular} \\ \hline
		\begin{tabular}[c]{@{}l@{}}is\_unbounded\_\\ array\end{tabular} &
		C++20 &
		\begin{tabular}[c]{@{}l@{}}检查是否为未知边界的数组类型(比如int{[}{]}).\end{tabular} \\ \hline
		is\_scoped\_enum &
		C++23 &
		是否为范围枚举类型。 \\ \hline
		\begin{tabular}[c]{@{}l@{}}has\_unique\_\\ object\_\\ representation\end{tabular} &
		C++17 &
		\begin{tabular}[c]{@{}l@{}}是否可复制,并且该类型的两个具有相同值的对象\\ 也具有相同表现的形式。\end{tabular} \\ \hline
	\end{tabular}
\end{table}

\begin{center}
表 5.2
\end{center}

其中大多数可能很容易理解,但有两个乍一看似乎是相同的,is\_trivial和\_trivial\_copyable。对于标量类型或标量类型数组,这两者都成立,也适用于可简单复制的类或此类数组,但is\_trivial仅适用于具有普通默认构造函数的可复制类。

根据C++20标准§11.4.4.1,若默认构造函数不是用户提供的,那么默认构造函数是普通的,类没有虚成员函数,没有虚基类,没有具有默认初始化式的非静态成员,其每个直接基类都有一个普通的默认构造函数,类的每个非静态成员也都有一个普通的默认构造函数。为了更好地理解这一点,来看看下面的例子:

\begin{lstlisting}[style=styleCXX]
struct foo
{
	int a;
};

struct bar
{
	int a = 0;
};

struct tar
{
	int a = 0;
	tar() : a(0) {}
};

std::cout << std::is_trivial_v<foo> << '\n'; // true
std::cout << std::is_trivial_v<bar> << '\n'; // false
std::cout << std::is_trivial_v<tar> << '\n'; // false

std::cout << std::is_trivially_copyable_v<foo>
          << '\n'; // true
std::cout << std::is_trivially_copyable_v<bar>
          << '\n'; // true
std::cout << std::is_trivially_copyable_v<tar>
          << '\n'; // true
\end{lstlisting}

本例中,有三个类似的类。这三个变量foo、bar和tar都是可复制的,只有foo类是一个普通类,因为它有一个普通的默认构造函数。bar类有一个带有默认初始化式的非静态成员,tar类有一个用户定义的构造函数,这使得它们不普通。

除了简单的复制能力之外,还可以在其他类型特征的帮助下查询其他操作。

\subsubsubsection{5.5.3\hspace{0.2cm}查询支持的操作}

下面的类型特征可以查询类型支持的操作:

\begin{table}[H]
\centering
	\begin{tabular}{|l|l|}
		\hline
		\textbf{名称} & \textbf{描述} \\ \hline
		\begin{tabular}[c]{@{}l@{}}is\_constructible\\ is\_trivially\_constructible\\ is\_nothrow\_constructible\end{tabular} &
		是否有可以接受特定参数的构造函数。 \\ \hline
		\begin{tabular}[c]{@{}l@{}}is\_default\_constructible\\ is\_trivially\_default\_constructible\\ is\_nothrow\_default\_constructible\end{tabular} &
		是否有默认构造函数。 \\ \hline
		\end{tabular}
\end{table}

\begin{table}[H]
\centering
	\begin{tabular}{|l|l|}
		\hline
		\textbf{名称} &
		\textbf{描述} \\ \hline
		\begin{tabular}[c]{@{}l@{}}is\_copy\_constructible\\ is\_tribially\_copy\_constructible\\ is\_nothrow\_copy\_constructible\end{tabular} &
		是否具有复制构造函数 \\ \hline
		\begin{tabular}[c]{@{}l@{}}is\_move\_constructible\\ is\_trivially\_move\_constructible\\ is\_nothrow\_move\_constructible\end{tabular} &
		是否具有移动构造函数。 \\ \hline
		\begin{tabular}[c]{@{}l@{}}is\_assignable\\ is\_trivially\_assignable\\ is\_nothrow\_assignable\end{tabular} &
		是否具有特定参数的赋值操作符。 \\ \hline
		\begin{tabular}[c]{@{}l@{}}is\_copy\_assignable\\ is\_trivially\_copy\_assignable\\ is\_nothrow\_copy\_assignable\end{tabular} &
		是否为复制赋值运算符。 \\ \hline
		\begin{tabular}[c]{@{}l@{}}is\_move\_assignable\\ is\_trivially\_move\_assigneable\\ is\_nothrow\_move\_assignable\end{tabular} &
		否有移动赋值操作符。 \\ \hline
		\begin{tabular}[c]{@{}l@{}}is\_destructible\\ is\_trivially\_destructible\\ is\_nothrow\_destructible\end{tabular} &
		是否具有未删除的析构函数。 \\ \hline
		has\_virtual\_destructor &
		是否具有虚析构函数。 \\ \hline
		\begin{tabular}[c]{@{}l@{}}is\_swappable\_with\\ is\_swappable\\ is\_nothrow\_swappable\_with\\ is\_nothrow\_swappable\end{tabular} &
		\begin{tabular}[c]{@{}l@{}}是否可以交换相同类型的对象或\\ 不同类型的对象。\end{tabular} \\ \hline
	\end{tabular}
\end{table}

\begin{center}
表 5.3
\end{center}

除了最后一个是在C++17引入,其他都在C++11引入。每种类型特征都有多个变体,包括用于检查普通操作或使用noexcept说明符声明为无异常抛出操作的变体。

现在来看看类型特征,如何查询类型之间的关系。

\subsubsubsection{5.5.4\hspace{0.2cm}查询类型的关系}

在这里,可以找到几个类型特征,可以查询类型之间的关系。这些类型特征如下所示:

\begin{table}[H]
\centering
	\begin{tabular}{|l|l|l|}
		\hline
		\textbf{名称} &
		\textbf{C++版本} &
		\textbf{描述} \\ \hline
		is\_same &
		C++11 &
		两个类型是否相同,包括可能的cv限定符。 \\ \hline
		is\_base\_of &
		C++11 &
		一个类型是否派生自另一个类型。 \\ \hline
		\begin{tabular}[c]{@{}l@{}}is\_convertible\\ is\_nothrow\_convertible\end{tabular} &
		\begin{tabular}[c]{@{}l@{}}C++11\\ C++14\end{tabular} &
		一种类型是否可以转换为另一种类型。 \\ \hline
		\begin{tabular}[c]{@{}l@{}}is\_invocable\\ is\_invocable\_r\\ is\_nothrow\_invocable\\ is\_nothrow\_invocable\_r\end{tabular} &
		C++17 &
		是否可以调用一个类型与指定参数类型。 \\ \hline
		is\_layout\_compatible &
		C++20 &
		\begin{tabular}[c]{@{}l@{}}检查两种类型是否具有兼容的布局。若两个类\\是相同类型  (忽略cv限定符),或者其公共初始\\ 序列包含所有非静态数据成员和位字段,或者是\\ 具有相同底层类型的枚举,则为布局兼容。\end{tabular} \\ \hline
		\begin{tabular}[c]{@{}l@{}}is\_pointer\_\\ inconvertible\_base \_of\end{tabular} &
		C++20 &
		是否为另一类型的指针不可转换基类。 \\ \hline
	\end{tabular}
\end{table}

\begin{center}
表 5.4
\end{center}

这些类型特征中,使用最多的可能是std::is\_same。这种类型特征在判断两种类型是否相同时非常有用,不过这种类型的特征需要考虑const和volatile限定符;因此,int和int const不是同一类型。

可以使用这个类型特征来扩展前面所示的as\_string函数的实现,若用true或false参数调用,则会输出1或0,而不是true/false。这里,可以为bool类型添加一个显式检查,并返回一个包含这两个值之一的字符串:

\begin{lstlisting}[style=styleCXX]
template <typename T>
std::string as_string(T value)
{
	if constexpr (std::is_null_pointer_v<T>)
		return "null";
	else if constexpr (std::is_same_v<T, bool>)
		return value ? "true" : "false";
	else if constexpr (std::is_arithmetic_v<T>)
		return std::to_string(value);
	else
		static_assert(always_false<T>);
}

std::cout << as_string(true) << '\n'; // prints true
std::cout << as_string(false) << '\n'; // prints false
\end{lstlisting}

目前看到的所有类型特征,都用于查询关于类型的某种信息。在下一节中,我们将看到对类型执行进行修改的类型特征。

\subsubsubsection{5.5.5\hspace{0.2cm}修改cv限定符、引用、指针或符号}

类型上执行转换的类型特征也称为元函数。这些类型特征提供了一个称为type的成员类型(typedef),表示转换后的类型。这类类型特征包括:

\begin{table}[H]
\centering
	\begin{tabular}{|l|l|}
		\hline
		\textbf{名称} &
		\textbf{描述} \\ \hline
		\begin{tabular}[c]{@{}l@{}}add\_cv\\ add\_const\\ add\_volatile\end{tabular} &
		将const、volatile或两者都添加到类型中。 \\ \hline
		\begin{tabular}[c]{@{}l@{}}remove\_cv\\ remove\_const\\ remove\_volatile\end{tabular} &
		从类型中删除const、volatile或同时删标识符。 \\ \hline
		\begin{tabular}[c]{@{}l@{}}add\_lvalue\_reference\\ add\_rvalue\_reference\end{tabular} &
		向类型添加左值或右值引用。 \\ \hline
		remove\_reference &
		从类型中删除引用(左值或右值)。 \\ \hline
		remove\_cvref &
		\begin{tabular}[c]{@{}l@{}}从类型中删除const和volatile限定符以及左值或右值引用。\\ 融合了remove\_cv和remove\_reference。\end{tabular} \\ \hline
		add\_pointer                                                                  & 添加指向类型的指针。                     \\ \hline
		remove\_pointer                                                               & 从类型中移除指针。                \\ \hline
		\begin{tabular}[c]{@{}l@{}}make\_signed\\ make\_unsigned\end{tabular} &
		\begin{tabular}[c]{@{}l@{}}创建一个有符号或无符号的整型(bool类型除外)或枚举类型。\\ 支持的整数类型有short, int, long, long long, char, wchar\_t, \\ char8\_t, char16\_t和 char32\_t。\end{tabular} \\ \hline
		\begin{tabular}[c]{@{}l@{}}remove\_extent\\ remove\_all\_extents\end{tabular} & 从数组类型中移除一个范围或所有范围。 \\ \hline
	\end{tabular}
\end{table}

\begin{center}
表 5.5
\end{center}

除了remove\_cvref是在C++20中添加的,本表中列出的所有其他类型特征在C++11中可用。这些并不是标准库中的所有元函数,更多的将在下一节中列出。

\subsubsubsection{5.5.6\hspace{0.2cm}各种转换}

除了前面列出的元函数之外,还有其他执行类型转换的类型特征。其中重要的如下表所示:

\begin{table}[H]
\centering
	\begin{tabular}{|l|l|l|}
		\hline
		\textbf{名称} &
		\textbf{C++版本} &
		\textbf{描述} \\ \hline
		enable\_if &
		C++11 &
		启用从重载解析中删除函数重载或模板特化。 \\ \hline
		conditional &
		C++11 &
		\begin{tabular}[c]{@{}l@{}}通过基于编译时布尔条件进行选择,\\ 将type成员类型定义为两种可能类型之一。\end{tabular} \\ \hline
		decay &
		C++11 &
		\begin{tabular}[c]{@{}l@{}}在类型上应用转换(数组类型为数组到指针,\\ 引用类型为左值到右值,函数类型为指针),\\ 删除const和volatile限定符,并使用结果成员\\ 类型定义类型作为其自己的成员类型定义类型。\end{tabular} \\ \hline
		common\_type &
		C++11 &
		从一组类型中确定公共类型。 \\ \hline
		common\_reference &
		C++20 &
		从一组类型中确定公共引用类型。 \\ \hline
		underlaying\_type &
		C++11 &
		确定枚举类型的基础类型。 \\ \hline
		void\_t &
		C++17 &
		将类型序列映射到void类型的类型别名。 \\ \hline
		type\_identity &
		C++20 &
		提供成员typedef类型作为类型参数T的别名。 \\ \hline
	\end{tabular}
\end{table}

列表中,已经讨论了enable\_if,还有一些需要举例说明的类型特征。先来看看std::decay,考虑一下as\_string函数的另一种实现方式:

\begin{lstlisting}[style=styleCXX]
template <typename T>
std::string as_string(T&& value)
{
	if constexpr (std::is_null_pointer_v<T>)
		return "null";
	else if constexpr (std::is_same_v<T, bool>)
		return value ? "true" : "false";
	else if constexpr (std::is_arithmetic_v<T>)
		return std::to_string(value);
	else
		static_assert(always_false<T>);
}
\end{lstlisting}

其中的变化是将参数传递给函数的方式。不是按值传递,而是按右值引用传递。若还记得第4章,这是一个转发引用。仍然可以通过传递右值(例如字面量)进行调用,但传递左值会触发编译器错误:

\begin{lstlisting}[style=styleCXX]
std::cout << as_string(true) << '\n'; // OK
std::cout << as_string(42) << '\n'; // OK

bool f = true;
std::cout << as_string(f) << '\n'; // error

int n = 42;
std::cout << as_string(n) << '\n'; // error
\end{lstlisting}

最后两个调用触发static\_assert语句失败。实际的类型模板参数是bool\&和int\&。因此std::is\_same<bool, bool\&>将用false初始化值member。类似地,std::is\_arithmetic<int\&>也会做同样的事情。为了求值这些类型,需要忽略引用以及const和volatile限定符。辅助我们的类型特征是std::decay,它会执行几个转换,如前一个表所述。其概念执行如下所示:

\begin{lstlisting}[style=styleCXX]
template <typename T>
struct decay
{
private:
	using U = typename std::remove_reference_t<T>;
public:
	using type = typename std::conditional_t<
		std::is_array_v<U>,
		typename std::remove_extent_t<U>*,
		typename std::conditional_t<
			std::is_function<U>::value,
			typename std::add_pointer_t<U>,
			typename std::remove_cv_t<U>
		>
	>;
};
\end{lstlisting}

从这段代码中,可以看到std::decay是在其他元函数的帮助下实现的,包括std::conditional,是基于编译时表达式在一种类型或另一种类型之间进行选择的关键。实际上,这种类型特征会多次使用,若需要根据多个条件进行选择可以这样做。

在std::decay的帮助下,可以修改as\_string函数、剥离引用和cv-限定符的实现:

\begin{lstlisting}[style=styleCXX]
template <typename T>
std::string as_string(T&& value)
{
	using value_type = std::decay_t<T>;
	
	if constexpr (std::is_null_pointer_v<value_type>)
		return "null";
	else if constexpr (std::is_same_v<value_type, bool>)
		return value ? "true" : "false";
	else if constexpr (std::is_arithmetic_v<value_type>)
		return std::to_string(value);
	else
		static_assert(always_false<T>);
}
\end{lstlisting}

通过修改实现,可使消除前面对as\_string的编译错误。

在std::decay的实现中,多次使用了std::conditional。这是一个相当好用的元函数,可以帮助简化许多实现。在第2章中,我们看到了一个例子,构建了一个名为list\_t的列表类型。它有一个名为type的成员别名模板,若列表的大小为1,则该模板类型为T,若列表的大小大于1,则该模板类型为std::vector<T>。再来回顾一下这个代码段:

\begin{lstlisting}[style=styleCXX]
template <typename T, size_t S>
struct list
{
	using type = std::vector<T>;
};

template <typename T>
struct list<T, 1>
{
	using type = T;
};

template <typename T, size_t S>
using list_t = typename list<T, S>::type;
\end{lstlisting}

在std::conditional的帮助下,这个实现可以大大简化:

\begin{lstlisting}[style=styleCXX]
template <typename T, size_t S>
using list_t =
	typename std::conditional<S ==
					1, T, std::vector<T>>::type;
\end{lstlisting}

没有必要依赖类模板特化来定义这样的列表类型,整个解决方案可以简化为定义一个别名模板。可以用一些static\_assert来验证它是否如预期的那样工作:

\begin{lstlisting}[style=styleCXX]
static_assert(std::is_same_v<list_t<int, 1>, int>);
static_assert(std::is_same_v<list_t<int, 2>,
							std::vector<int>>);
\end{lstlisting}

举例说明每个标准类型特征的使用超出了本书的范畴。然而,本章的下一节提供了更复杂的例子,需要使用几个标准的类型特征。



























