在<utility>中文件中,std::declval是一个工具类型的操作函数。与已经了解的std::move和std::forward等函数属于同一类别,其功能非常简单:将右值引用添加到类型模板参数中。这个函数的声明如下所示:

\begin{lstlisting}[style=styleCXX]
template<class T>
typename std::add_rvalue_reference<T>::type declval() noexcept;
\end{lstlisting}

这个函数没有定义,因此不能直接调用,只能在未求值的上下文中使用——decltype、sizeof、typeid和noexcept。这些是仅在编译时执行的上下文,在运行时不会进行计算。std::declval的目的是帮助对没有默认构造函数或有默认构造函数,但由于是private或protected而不能访问的类型,进行依赖类型求值。

为了理解这是如何工作的,来看一个类模板,它将两个不同类型的值组合在一起,我们希望为这两个类型的值使用加号操作符的结果,创建一个类型别名。如何定义这样的类型别名?看看下面的代码:

\begin{lstlisting}[style=styleCXX]
template <typename T, typename U>
struct composition
{
	using result_type = decltype(???);
};
\end{lstlisting}

可以使用decltype说明符,但需要提供一个表达式。不能是decltype(T + U),因为它们是类型,而不是值。可以调用默认构造函数,因此可以使用表达式decltype(T\{\} + U\{\})。这可以很好地用于内置类型,如int和double,如下面的代码段所示:

\begin{lstlisting}[style=styleCXX]
static_assert(
	std::is_same_v<double,
		composition<int, double>::result_type>);
\end{lstlisting}

也适用于具有(可访问的)默认构造函数的类型,但不能用于没有默认构造函数的类型。以下的wrapper就是这样一个例子:

\begin{lstlisting}[style=styleCXX]
struct wrapper
{
	wrapper(int const v) : value(v){}
	int value;
	
	friend wrapper operator+(int const a, wrapper const& w)
	{
		return wrapper(a + w.value);
	}

	friend wrapper operator+(wrapper const& w, int const a)
	{
		return wrapper(a + w.value);
	}
};

// error, no appropriate default constructor available
static_assert(
	std::is_same_v<wrapper,
		composition<int,wrapper>::result_type>);
\end{lstlisting}

这里的解决方案是使用std::declval()。类模板组合的实现将进行如下修改:

\begin{lstlisting}[style=styleCXX]
template <typename T, typename U>
struct composition
{
	using result_type = decltype(std::declval<T>() +
								 std::declval<U>());
};
\end{lstlisting}

修改之后,前面显示的两个静态断言都可以编译,没有任何错误。这个函数避免需要使用特定的值来确定表达式的类型。它生成一个类型为T的值,而不涉及默认构造函数。返回右值引用的原因是,能够处理函数不能返回的类型,例如:数组和抽象类型。

前面wrapper类的定义包含两个友元操作符。当涉及到模板时,“友情”有一定的特殊性。我们将在下一节讨论这个问题。

















