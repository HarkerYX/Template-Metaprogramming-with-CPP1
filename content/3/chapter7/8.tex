This chapter was dedicated to learning various metaprogramming techniques. We started by understanding the differences between dynamic and static polymorphism and then looked at the curiously recurring template pattern for implementing the latter.

Mixins was another pattern that has a similar purpose as CRTP—adding functionality to classes, but unlike CRTP, without modifying them. The third technique we learned about was type erasure, which allows similar types that are unrelated to be treated generically. In the second part, we learned about tag dispatching – which allow us to select between overloads at compile time, and expression templates – which enable lazy evaluation of a computation at compile-time to avoid inefficient operations that occur at runtime.
Lastly, we explored typelists and learned how they are used and how we can implement operations with them.

In the next chapter, we will look at the core pillars of the standard template library, containers, iterators, and algorithms.