本章致力于学习各种元编程技术。我们首先了解动态多态和静态多态之间的区别,然后研究了用于实现静态多态的奇异迭代模板模式。

混入(Mixins)是另一种与CRTP目的相似的模式——向类中添加功能,但与CRTP不同的是,不需要修改它们。我们学习的第三种技术是类型擦除,允许对不相关的相似类型进行泛型处理。第二部分中,我们学习了标记分派——允许在编译时的重载和表达式模板之间进行选择——支持在编译时对计算进行惰性求值,以避免在运行时发生低效操作。最后,我们探讨了类型列表,并学习了如何使用它们,以及如何使用它们实现操作。

下一章中,我们将讨论标准模板库、容器、迭代器和算法的核心。