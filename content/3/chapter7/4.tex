类型擦除描述了一种模式,其中删除类型信息,允许以通用方式处理不一定相关的类型。这不是C++语言特有的东西,这个概念也存在于其他语言(如Python和Java)。有不同形式的类型擦除,如多态性和使用void指针(C语言的使用方式,需要避免),但真正的类型擦除是通过模板实现的。讨论这个问题之前,先简要地看一下其他问题。

类型擦除的最基本形式是使用void指针。这是典型的C语言方式,尽管在C++中可以,但绝不推荐这样做。这样做不是类型安全的,因此很容易出错。但为了便于讨论,来看看这种方法。

再次假设我们有knight和mage类型,他们都有攻击功能(行为),我们希望以一种共同的方式来对待他们,以展示这种行为:

\begin{lstlisting}[style=styleCXX]
struct knight
{
	void attack() { std::cout << "draw sword\n"; }
};

struct mage
{
	void attack() { std::cout << "spell magic curse\n"; }
};
\end{lstlisting}

类C语言的实现中,可以为每一种类型都创建一个函数,对该类型的对象使用void*,将其转换为预期的指针类型,然后调用attack成员函数:

\begin{lstlisting}[style=styleCXX]
void fight_knight(void* k)
{
	reinterpret_cast<knight*>(k)->attack();
}

void fight_mage(void* m)
{
	reinterpret_cast<mage*>(m)->attack();
}
\end{lstlisting}

它们有相似的特征,唯一不同的是名字。可以定义一个函数指针,然后将一个对象(或者更准确地说,一个指向对象的指针)与一个指向处理它的正确函数的指针关联起来。以下是具体实现:

\begin{lstlisting}[style=styleCXX]
using fight_fn = void(*)(void*);
void fight(
	std::vector<std::pair<void*, fight_fn>> const& units)
{
	for (auto& u : units)
	{
		u.second(u.first);
	}
}
\end{lstlisting}

最后一个代码片段中没有关于类型的信息,所有这些都已使用void指针擦除。fight函数可以按如下方式调用:

\begin{lstlisting}[style=styleCXX]
knight k;
mage m;

std::vector<std::pair<void*, fight_fn>> units {
	{&k, &fight_knight},
	{&m, &fight_mage},
};

fight(units);
\end{lstlisting}

C++的角度来看,这可能看起来很奇怪。这个示例中,我将C与C++类相结合,希望不会在生产环境中看到这样的代码片段。若将一个mage传递给fight\_knight函数或者相反,就会因为一个简单的输入错误而出错。然而,这是可能的,并且是类型擦除的一种形式。C++中一个明显的替代解决方案是通过继承使用多态性。

这是在本章开头看到的第一个解决方案:

\begin{lstlisting}[style=styleCXX]
struct game_unit
{
	virtual void attack() = 0;
};

struct knight : game_unit
{
	void attack() override
	{ std::cout << "draw sword\n"; }
};

struct mage : game_unit
{
	void attack() override
	{ std::cout << "spell magic curse\n"; }
};

void fight(std::vector<game_unit*> const & units)
{
	for (auto unit : units)
		unit->attack();
}
\end{lstlisting}

fight功能可以同时处理knight和mage的对象,不知道传递给它地址的实际对象(在vector内),因此可以说类型并没有完全删除。knight和mage都是game\_unit,而fight函数处理任何game\_unit。对于这个函数要处理的另一种类型,需要派生自game\_unit纯抽象类。

有时候这是不可能的。在类似的事情中处理不相关的类型(鸭子类型的过程),但不能改变这些类型。例如,我们并不拥有源代码。这个问题的解决方案,就是使用模板进行真正的类型擦除。

看到这个模式是什么样子之前,来一步一步理解这个模式的发展,从不相关的knight和mage开始,以及不能修改他们的实现作为前提。但我们可以编写包装器,为公共功能(行为)提供统一的接口:

\begin{lstlisting}[style=styleCXX]
struct knight
{
	void attack() { std::cout << "draw sword\n"; }
};

struct mage
{
	void attack() { std::cout << "spell magic curse\n"; }
};

struct game_unit
{
	virtual void attack() = 0;
	virtual ~game_unit() = default;
};

struct knight_unit : game_unit
{
	knight_unit(knight& u) : k(u) {}
	void attack() override { k.attack(); }
	
private:
	knight& k;
};

struct mage_unit : game_unit
{
	mage_unit(mage& u) : m(u) {}
	void attack() override { m.attack(); }
	
private:
	mage& m;
};

void fight(std::vector<game_unit*> const & units)
{
	for (auto u : units)
	u->attack();
}
\end{lstlisting}

我们不需要像在knight和mage中那样调用game\_unit中的attack成员函数,其名称可以随意,纯粹是基于模仿原始行为名称。fight函数接受一个指向game\_unit的指针集合,因此能够同时处理knight和mage对象:

\begin{lstlisting}[style=styleCXX]
knight k;
mage m;

knight_unit ku{ k };
mage_unit mu{ m };

std::vector<game_unit*> v{ &ku, &mu };
fight(v);
\end{lstlisting}

这个解决方案的问题是有很多重复的代码,knight\_unit和mage\_unit基本上是一样的。当其他类需要类似地处理时,这种重复会增加更多。代码复制的解决方案是使用模板。我们用下面的职业模板替换knight\_unit和mage\_unit:

\begin{lstlisting}[style=styleCXX]
template <typename T>
struct game_unit_wrapper : public game_unit
{
	game_unit_wrapper(T& unit) : t(unit) {}
	
	void attack() override { t.attack(); }
private:
	T& t;
};
\end{lstlisting}

这个类在源代码中只有一个副本,但是编译器会使用实例化多个特化。除某些类型限制外,类型信息都会擦除——T类型必须有一个名为attack的成员函数,该函数不接受参数。注意,fight函数根本没有改变。调用端代码需要稍作修改:

\begin{lstlisting}[style=styleCXX]
knight k;
mage m;

game_unit_wrapper ku{ k };
game_unit_wrapper mu{ m };

std::vector<game_unit*> v{ &ku, &mu };
fight(v);
\end{lstlisting}

这将我们引向类型擦除模式的形式,将抽象基类和包装器类模板放在另一个类中:

\begin{lstlisting}[style=styleCXX]
struct game
{
	struct game_unit
	{
		virtual void attack() = 0;
		virtual ~game_unit() = default;
	};

	template <typename T>
	struct game_unit_wrapper : public game_unit
	{
		game_unit_wrapper(T& unit) : t(unit) {}
		
		void attack() override { t.attack(); }
	private:
		T& t;
	};

	template <typename T>
	void addUnit(T& unit)
	{
		units.push_back(
		std::make_unique<game_unit_wrapper<T>>(unit));
	}

	void fight()
	{
		for (auto& u : units)
			u->attack();
	}
private:
	std::vector<std::unique_ptr<game_unit>> units;
};
\end{lstlisting}

game类包含game\_unit对象的集合,并有一个向game\_unit(具有attack成员函数)添加新包装器的方法。还有一个成员函数fight,用于调用常见的行为。这次的调用端的代码如下所示:

\begin{lstlisting}[style=styleCXX]
knight k;
mage m;

game g;
g.addUnit(k);
g.addUnit(m);

g.fight();
\end{lstlisting}

类型擦除模式中,抽象基类称为概念,继承的包装器称为模型。若要以既定的方式实现类型擦除模式,可以进行如下实现:

\begin{lstlisting}[style=styleCXX]
struct unit
{
	template <typename T>
	unit(T&& obj) :
		unit_(std::make_shared<unit_model<T>>(
				std::forward<T>(obj)))
	{}
	
	void attack()
	{
		unit_->attack();
	}

	struct unit_concept
	{
		virtual void attack() = 0;
		virtual ~unit_concept() = default;
	};

	template <typename T>
	struct unit_model : public unit_concept
	{
		unit_model(T& unit) : t(unit) {}
		
		void attack() override { t.attack(); }
	private:
		T& t;
	};

private:
	std::shared_ptr<unit_concept> unit_;
};

void fight(std::vector<unit>& units)
{
	for (auto& u : units)
		u.attack();
}
\end{lstlisting}

这段代码片中,game\_unit重命名为unit\_concept,game\_unit\_wrapper重命名为unit\_model,除了名字没有其他变化。它们是名为unit的新类的成员,该类存储一个指针,指向实现unit\_concept的对象;这可以是unit\_model<knight>或unit\_model<mage>。unit类有一个模板构造函数,能够从knight和mage对象中创建模型对象。

其还有一个公共成员函数attack(同样,它可以有任何名称)。另一方面,fight函数处理unit对象,并调用它们的fight成员函数。调用端代码可能如下所示:

\begin{lstlisting}[style=styleCXX]
knight k;
mage m;

std::vector<unit> v{ unit(k), unit(m) };

fight(v);
\end{lstlisting}

若想知道这个模式在实际代码中现在哪里使用,标准库中就有两个例子:

\begin{itemize}
\item
std::function: 这是一个通用的多态函数包装器,能够存储、复制和调用可调用的东西,例如函数、Lambda表达式、绑定表达式、函数对象、指向成员函数的指针和指向数据成员的指针。下面是一个使用std::function的例子:

\begin{lstlisting}[style=styleCXX]
class async_bool
{
	std::function<bool()> check;
public:
	async_bool() = delete;
	async_bool(std::function<bool()> checkIt)
		: check(checkIt)
	{ }
	
	async_bool(bool val)
		: check([val]() {return val; })
	{ }
	
	operator bool() const { return check(); }
};

async_bool b1{ false };
async_bool b2{ true };
async_bool b3{ []() { std::cout << "Y/N? ";
					  char c; std::cin >> c;
					  return c == 'Y' || c == 'y'; } };
				  
if (b1) { std::cout << "b1 is true\n"; }
if (b2) { std::cout << "b2 is true\n"; }
if (b3) { std::cout << "b3 is true\n"; }
\end{lstlisting}

\item
std::any: 这是一个将容器表示为可复制构造类型值的类。下面的代码中使用了一个例子:

\begin{lstlisting}[style=styleCXX]
std::any u;

u = knight{};
if (u.has_value())
	std::any_cast<knight>(u).attack();

u = mage{};
if (u.has_value())
	std::any_cast<mage>(u).attack();
\end{lstlisting}
\end{itemize}

类型擦除是一种习语,将面向对象编程的继承与模板结合起来,以创建可以存储任何类型的包装器。本节中,我们了解了模式的表现形式和工作方式,以及该模式的一些实际实现。

下一节中,我们将讨论标记分派。



