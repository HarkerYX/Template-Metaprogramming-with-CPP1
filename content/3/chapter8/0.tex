By reaching this point of the book, you have learned everything about the syntax and the mechanism of templates in C++, up to the latest version of the standard, C++20. This has equipped you with the necessary knowledge to write templates from simple forms to complex ones. Templates are the key to writing generic libraries. Even though you might not write such a library yourself, you’d still be using one or more. In fact, the everyday code that you’re writing in C++ uses templates. And the main reason for that is that as a modern C++ developer, you’re using the standard library, which is a library based on templates.

However, the standard library is a collection of many libraries, such as the containers library, iterators library, algorithms library, numeric library, input/output library, filesystem library, regular expressions library, thread support library, utility libraries, and others. Overall, it’s a large library that could make the topic of at least an entire book. However, it is worth exploring some key parts of the library to help you get a better understanding of some of the concepts and types you are or could be using regularly.

Because addressing this topic in a single chapter would lead to a significantly large chapter, we will split the discussion into two parts. In this chapter, we will address the following topics:

\begin{itemize}
\item
Understanding the design of containers, iterators, and algorithms

\item
Creating a custom container and iterator

\item
Writing a custom general-purpose algorithm
\end{itemize}

By the end of this chapter, you will have a good understanding of the three main pillars of the standard template library, which are containers, iterators, and algorithms.

We will begin this chapter with an overview of what the standard library has to offer in this respect.


