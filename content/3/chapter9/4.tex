标准库包含一系列范围适配器,可用于解决许多不同的任务。新版本的标准中增加了更多的内容。但某些情况下,可能希望创建自己的范围适配器,以便与范围库中的其他适配器一起使用。这实际上不是一项微不足道的任务,所以在本章的最后一节中,我们将探讨编写这样一个范围适配器所需要遵循的步骤。

为此,考虑一个范围适配器,接受范围的第n个元素,而跳过其他元素。我们将这个适配器称为step\_view。我们可以使用它来编写如下代码:

\begin{lstlisting}[style=styleCXX]
for (auto i : std::views::iota(1, 10) | views::step(1))
	std::cout << i << '\n';

for (auto i : std::views::iota(1, 10) | views::step(2))
	std::cout << i << '\n';

for (auto i : std::views::iota(1, 10) | views::step(3))
	std::cout << i << '\n';

for (auto i : std::views::iota(1, 10) | views::step(2) | 	
				std::views::take(3))
	std::cout << i << '\n';
\end{lstlisting}

第一个循环将打印从1到9的所有数字。第二个循环将输出所有的奇数,1、3、5、7、9。第三个循环将输出1,4,7。最后,第四个循环将输出1,3,5。

为了实现这个,我们需要以下实现:

\begin{itemize}
\item
定义范围适配器的类模板

\item
推导指南,用于帮助演绎范围适配器的类模板参数

\item
为范围适配器定义迭代器类型的类模板

\item
为范围适配器定义哨点类型的类模板

\item
实现所需的重载管道操作符(|)和辅助函子

\item
编译时常量全局对象,以简化范围适配器的使用
\end{itemize}

一个个地来了解如何定义它们,从哨兵类开始。哨点是对过尾迭代器的抽象,允许检查迭代是否到达了范围的末端。哨兵可以使结束迭代器的类型与范围迭代器的类型不同,哨兵不能解除引用或增加。下面是其定义:

\begin{lstlisting}[style=styleCXX]
template <typename R>
struct step_iterator;

template <typename R>
struct step_sentinel
{
	using base = std::ranges::iterator_t<R>;
	using size_type = std::ranges::range_difference_t<R>;
	
	step_sentinel() = default;
	
	constexpr step_sentinel(base end) : end_{ end } {}
	constexpr bool is_at_end(step_iterator<R> it) const;
	
private:
	base end_;
};

// definition of the step_iterator type

template <typename R>
constexpr bool step_sentinel<R>::is_at_end(
	step_iterator<R> it) const
{
	return end_ == it.value();
}
\end{lstlisting}

哨兵由迭代器构造,包含一个名为is\_at\_end的成员函数,该函数检查存储的范围迭代器是否等于存储在step\_iterator对象中的范围迭代器。这个类型step\_iterator是一个类模板,定义了范围适配器的迭代器类型,称之为step\_view。下面是该迭代器类型的实现:

\begin{lstlisting}[style=styleCXX]
template <typename R>
struct step_iterator : std::ranges::iterator_t<R>
{
	using base
		= std::ranges::iterator_t<R>;
	using value_type
		= typename std::ranges::range_value_t<R>;
	using reference_type
		= typename std::ranges::range_reference_t<R>;
		
	constexpr step_iterator(
		base start, base end,
		std::ranges::range_difference_t<R> step) :
		pos_{ start }, end_{ end }, step_{ step }
	{
	}

	constexpr step_iterator operator++(int)
	{
		auto ret = *this;
		pos_ = std::ranges::next(pos_, step_, end_);
		return ret;
	}

	constexpr step_iterator& operator++()
	{
		pos_ = std::ranges::next(pos_, step_, end_);
		return *this;
	}

	constexpr reference_type operator*() const
	{
		return *pos_;
	}
	constexpr bool operator==(step_sentinel<R> s) const
	{
		return s.is_at_end(*this);
	}

	constexpr base const value() const { return pos_; }
	
private:
	base pos_;
	base end_;
	std::ranges::range_difference_t<R> step_;
};
\end{lstlisting}

这个类型必须有几个成员:

\begin{itemize}
\item
别名模板base,表示基础范围迭代器的类型。

\item
别名模板称为value\_type,表示底层范围的元素类型。

\item
重载操作符++和*。

\item
重载操作符==将该对象与哨兵进行比较。
\end{itemize}

++操作符的实现使用std::ranges::next约束算法对迭代器增加N个位置,但不超过范围的结束。

为了将step\_iterator和step\_sentinel对用于step\_view范围适配器,必须确保这个对实际上是格式良好的。为此,必须确保step\_iterator类型是一个输入迭代器,并且step\_sentinel类型确实是step\_iterator类型的哨兵类型。这可以在static\_assert帮助下完成:

\begin{lstlisting}[style=styleCXX]
namespace details
{
	using test_range_t =
		std::ranges::views::all_t<std::vector<int>>;
	static_assert(
		std::input_iterator<step_iterator<test_range_t>>);
	static_assert(
		std::sentinel_for<step_sentinel<test_range_t>,
		step_iterator<test_range_t>>);
}
\end{lstlisting}

step\_iterator类型用于step\_view范围适配器的实现:

\begin{lstlisting}[style=styleCXX]
template<std::ranges::view R>
struct step_view :
	public std::ranges::view_interface<step_view<R>>
{
private:
	R base_;
	std::ranges::range_difference_t<R> step_;
	
public:
	step_view() = default;
	
	constexpr step_view(
		R base,
		std::ranges::range_difference_t<R> step)
			: base_(std::move(base))
			, step_(step)
	{
	}

	constexpr R base() const&
		requires std::copy_constructible<R>
	{ return base_; }
	constexpr R base()&& { return std::move(base_); }
	
	constexpr std::ranges::range_difference_t<R> const&
	increment() const
	{ return step_; }
	
	constexpr auto begin()
	{
		return step_iterator<R const>(
			std::ranges::begin(base_),
			std::ranges::end(base_), step_);
	}

	constexpr auto begin() const
	requires std::ranges::range<R const>
	{
		return step_iterator<R const>(
			std::ranges::begin(base_),
			std::ranges::end(base_), step_);
	}

	constexpr auto end()
	{
		return step_sentinel<R const>{
			std::ranges::end(base_) };
	}

	constexpr auto end() const
	requires std::ranges::range<R const>
	{
		return step_sentinel<R const>{
			std::ranges::end(base_) };
	}

	constexpr auto size() const
	requires std::ranges::sized_range<R const>
	{
		auto d = std::ranges::size(base_);
		return step_ == 1 ? d :
			static_cast<int>((d + 1)/step_); 
	}
	
	constexpr auto size()
	requires std::ranges::sized_range<R>
	{
		auto d = std::ranges::size(base_);
		return step_ == 1 ? d :
			static_cast<int>((d + 1)/step_);
	}
};
\end{lstlisting}

定义范围适配器时,必须遵循一个模式。这种模式表现为以下几个方面:

\begin{itemize}
\item
类模板必须有符合std::ranges::view概念的模板参数。

\item
类模板应该派生自std::ranges:view\_interface,其本身接受一个模板参数,应该是范围适配器类。这基本上是第7章中的CRTP实现。

\item
类必须有默认构造函数。

\item
类必须具有返回基础范围的基成员函数。

\item
该类必须有begin成员函数,该函数返回指向范围内第一个元素的迭代器。

\item
该类必须有一个结束成员函数,该函数返回一个指向范围中倒数第一元素的迭代器或一个哨兵。

\item
对于满足std::ranges::size\_range概念要求的范围,该类还必须包含名为size的成员函数,该函数返回范围内的元素数量。
\end{itemize}

为了能够对step\_view类使用类模板参数推导,应该定义一个用户定义的推导指南。这在第4中讨论过,指南应该如下所示:

\begin{lstlisting}[style=styleCXX]
template<class R>
step_view(R&& base,
		  std::ranges::range_difference_t<R> step)
	-> step_view<std::ranges::views::all_t<R>>;
\end{lstlisting}

为了能够使用管道迭代器(|)将此范围适配器与其他范围适配器组合在一起,必须重载此操作符。但需要一些辅助函数对象,如下所示:

\begin{lstlisting}[style=styleCXX]
namespace details
{
	struct step_view_fn_closure
	{
		std::size_t step_;
		constexpr step_view_fn_closure(std::size_t step)
			: step_(step)
		{
		}
	
		template <std::ranges::range R>
		constexpr auto operator()(R&& r) const
		{
			return step_view(std::forward<R>(r), step_);
		}
	};
	template <std::ranges::range R>
	constexpr auto operator | (R&& r,
								step_view_fn_closure&& a)
	{
		return std::forward<step_view_fn_closure>(a)(
			std::forward<R>(r));
	}
}
\end{lstlisting}

step\_view\_fn\_closure类是一个函数对象,存储了一个表示每个迭代器要跳过的元素数量的值。其重载调用操作符以一个范围作为参数,并返回从该范围创建的step\_view对象和要跳转的步数的值。

最后,希望能够以与标准库中可用的方式类似的方式编写代码,标准库为每个存在的范围适配器在std::views命名空间中提供了一个编译时全局对象。例如,可以使用std::views::transform,而不是std::ranges::transform\_view。类似地,我们希望有一个对象views::step,而不是step\_view(某个命名空间中)。为此,还需要另一个函数对象:

\begin{lstlisting}[style=styleCXX]
namespace details
{
	struct step_view_fn
	{
		template<std::ranges::range R>
		constexpr auto operator () (R&& r,
								    std::size_t step) const
		{
			return step_view(std::forward<R>(r), step);
		}
	
		constexpr auto operator () (std::size_t step) const
		{
			return step_view_fn_closure(step);
		}
	};
}

namespace views
{
	inline constexpr details::step_view_fn step;
}
\end{lstlisting}

step\_view\_fn类型是一个函数对象,对调用操作符有两个重载:一个接受一个范围和一个整数,并返回一个step\_view对象,另一个接受一个整数并返回这个值的闭包。更准确地说,是前面step\_view\_fn\_closure的实例。

实现了这些之后,就可以成功地运行本节开头所示的代码。我们已经完成了一个简单的范围适配器的实现,希望这能让您了解写入范围适配器需要什么。从细节上看,范围库非常复杂。本章中,已经了解了一些关于库内容的基础知识,如何简化代码,以及如何使用自定义特性扩展它。若想利用其他资源了解更多,这些知识应该只能算是一个起点。






























